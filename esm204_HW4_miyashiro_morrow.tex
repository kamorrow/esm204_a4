\PassOptionsToPackage{unicode=true}{hyperref} % options for packages loaded elsewhere
\PassOptionsToPackage{hyphens}{url}
%
\documentclass[]{article}
\usepackage{lmodern}
\usepackage{amssymb,amsmath}
\usepackage{ifxetex,ifluatex}
\usepackage{fixltx2e} % provides \textsubscript
\ifnum 0\ifxetex 1\fi\ifluatex 1\fi=0 % if pdftex
  \usepackage[T1]{fontenc}
  \usepackage[utf8]{inputenc}
  \usepackage{textcomp} % provides euro and other symbols
\else % if luatex or xelatex
  \usepackage{unicode-math}
  \defaultfontfeatures{Ligatures=TeX,Scale=MatchLowercase}
\fi
% use upquote if available, for straight quotes in verbatim environments
\IfFileExists{upquote.sty}{\usepackage{upquote}}{}
% use microtype if available
\IfFileExists{microtype.sty}{%
\usepackage[]{microtype}
\UseMicrotypeSet[protrusion]{basicmath} % disable protrusion for tt fonts
}{}
\IfFileExists{parskip.sty}{%
\usepackage{parskip}
}{% else
\setlength{\parindent}{0pt}
\setlength{\parskip}{6pt plus 2pt minus 1pt}
}
\usepackage{hyperref}
\hypersetup{
            pdftitle={A Climate Change Model},
            pdfauthor={Bobby Miyashiro \& Keene Morrow},
            pdfborder={0 0 0},
            breaklinks=true}
\urlstyle{same}  % don't use monospace font for urls
\usepackage[margin=1in]{geometry}
\usepackage{color}
\usepackage{fancyvrb}
\newcommand{\VerbBar}{|}
\newcommand{\VERB}{\Verb[commandchars=\\\{\}]}
\DefineVerbatimEnvironment{Highlighting}{Verbatim}{commandchars=\\\{\}}
% Add ',fontsize=\small' for more characters per line
\usepackage{framed}
\definecolor{shadecolor}{RGB}{248,248,248}
\newenvironment{Shaded}{\begin{snugshade}}{\end{snugshade}}
\newcommand{\AlertTok}[1]{\textcolor[rgb]{0.94,0.16,0.16}{#1}}
\newcommand{\AnnotationTok}[1]{\textcolor[rgb]{0.56,0.35,0.01}{\textbf{\textit{#1}}}}
\newcommand{\AttributeTok}[1]{\textcolor[rgb]{0.77,0.63,0.00}{#1}}
\newcommand{\BaseNTok}[1]{\textcolor[rgb]{0.00,0.00,0.81}{#1}}
\newcommand{\BuiltInTok}[1]{#1}
\newcommand{\CharTok}[1]{\textcolor[rgb]{0.31,0.60,0.02}{#1}}
\newcommand{\CommentTok}[1]{\textcolor[rgb]{0.56,0.35,0.01}{\textit{#1}}}
\newcommand{\CommentVarTok}[1]{\textcolor[rgb]{0.56,0.35,0.01}{\textbf{\textit{#1}}}}
\newcommand{\ConstantTok}[1]{\textcolor[rgb]{0.00,0.00,0.00}{#1}}
\newcommand{\ControlFlowTok}[1]{\textcolor[rgb]{0.13,0.29,0.53}{\textbf{#1}}}
\newcommand{\DataTypeTok}[1]{\textcolor[rgb]{0.13,0.29,0.53}{#1}}
\newcommand{\DecValTok}[1]{\textcolor[rgb]{0.00,0.00,0.81}{#1}}
\newcommand{\DocumentationTok}[1]{\textcolor[rgb]{0.56,0.35,0.01}{\textbf{\textit{#1}}}}
\newcommand{\ErrorTok}[1]{\textcolor[rgb]{0.64,0.00,0.00}{\textbf{#1}}}
\newcommand{\ExtensionTok}[1]{#1}
\newcommand{\FloatTok}[1]{\textcolor[rgb]{0.00,0.00,0.81}{#1}}
\newcommand{\FunctionTok}[1]{\textcolor[rgb]{0.00,0.00,0.00}{#1}}
\newcommand{\ImportTok}[1]{#1}
\newcommand{\InformationTok}[1]{\textcolor[rgb]{0.56,0.35,0.01}{\textbf{\textit{#1}}}}
\newcommand{\KeywordTok}[1]{\textcolor[rgb]{0.13,0.29,0.53}{\textbf{#1}}}
\newcommand{\NormalTok}[1]{#1}
\newcommand{\OperatorTok}[1]{\textcolor[rgb]{0.81,0.36,0.00}{\textbf{#1}}}
\newcommand{\OtherTok}[1]{\textcolor[rgb]{0.56,0.35,0.01}{#1}}
\newcommand{\PreprocessorTok}[1]{\textcolor[rgb]{0.56,0.35,0.01}{\textit{#1}}}
\newcommand{\RegionMarkerTok}[1]{#1}
\newcommand{\SpecialCharTok}[1]{\textcolor[rgb]{0.00,0.00,0.00}{#1}}
\newcommand{\SpecialStringTok}[1]{\textcolor[rgb]{0.31,0.60,0.02}{#1}}
\newcommand{\StringTok}[1]{\textcolor[rgb]{0.31,0.60,0.02}{#1}}
\newcommand{\VariableTok}[1]{\textcolor[rgb]{0.00,0.00,0.00}{#1}}
\newcommand{\VerbatimStringTok}[1]{\textcolor[rgb]{0.31,0.60,0.02}{#1}}
\newcommand{\WarningTok}[1]{\textcolor[rgb]{0.56,0.35,0.01}{\textbf{\textit{#1}}}}
\usepackage{graphicx,grffile}
\makeatletter
\def\maxwidth{\ifdim\Gin@nat@width>\linewidth\linewidth\else\Gin@nat@width\fi}
\def\maxheight{\ifdim\Gin@nat@height>\textheight\textheight\else\Gin@nat@height\fi}
\makeatother
% Scale images if necessary, so that they will not overflow the page
% margins by default, and it is still possible to overwrite the defaults
% using explicit options in \includegraphics[width, height, ...]{}
\setkeys{Gin}{width=\maxwidth,height=\maxheight,keepaspectratio}
\setlength{\emergencystretch}{3em}  % prevent overfull lines
\providecommand{\tightlist}{%
  \setlength{\itemsep}{0pt}\setlength{\parskip}{0pt}}
\setcounter{secnumdepth}{0}
% Redefines (sub)paragraphs to behave more like sections
\ifx\paragraph\undefined\else
\let\oldparagraph\paragraph
\renewcommand{\paragraph}[1]{\oldparagraph{#1}\mbox{}}
\fi
\ifx\subparagraph\undefined\else
\let\oldsubparagraph\subparagraph
\renewcommand{\subparagraph}[1]{\oldsubparagraph{#1}\mbox{}}
\fi

% set default figure placement to htbp
\makeatletter
\def\fps@figure{htbp}
\makeatother

\usepackage{etoolbox}
\makeatletter
\providecommand{\subtitle}[1]{% add subtitle to \maketitle
  \apptocmd{\@title}{\par {\large #1 \par}}{}{}
}
\makeatother
\usepackage{booktabs}
\usepackage{longtable}
\usepackage{array}
\usepackage{multirow}
\usepackage{wrapfig}
\usepackage{float}
\usepackage{colortbl}
\usepackage{pdflscape}
\usepackage{tabu}
\usepackage{threeparttable}
\usepackage{threeparttablex}
\usepackage[normalem]{ulem}
\usepackage{makecell}
\usepackage{xcolor}

\title{A Climate Change Model}
\providecommand{\subtitle}[1]{}
\subtitle{ESM 204: Homework \#4}
\author{Bobby Miyashiro \& Keene Morrow}
\date{5/27/2020}

\begin{document}
\maketitle

ESM 204: Homework \#4 (A Climate Change Model) Professor Christopher
Costello Due May 27, 8:00 am

This problem set asks you to build a climate-economy model with risk and
discounting and to conduct sensitivity analysis on it. Hint: I strongly
recommend using R and building ``functions'' for each of the equations
you see below. Under BAU, let tau(t) be the temperature in year t (t =
0, 1, \ldots{}, 200) relative to the temperature at time 0. Suppose

tau(t) = min(Tt/100, T) (1)

\begin{Shaded}
\begin{Highlighting}[]
\CommentTok{# Relative temperature change by year}
\NormalTok{rel_temp_t <-}\StringTok{ }\ControlFlowTok{function}\NormalTok{(T, year)\{}
\NormalTok{  temp_t =}\StringTok{ }\KeywordTok{ifelse}\NormalTok{(year }\OperatorTok{<=}\StringTok{ }\DecValTok{100}\NormalTok{, year }\OperatorTok{*}\StringTok{ }\NormalTok{(T }\OperatorTok{/}\StringTok{ }\DecValTok{100}\NormalTok{), T)}
  \KeywordTok{return}\NormalTok{(temp_t)}
\NormalTok{\}}

\NormalTok{tau_t <-}\StringTok{ }\ControlFlowTok{function}\NormalTok{(Tt, T)\{}
\NormalTok{  rel_temp =}\StringTok{ }\KeywordTok{min}\NormalTok{(Tt }\OperatorTok{/}\StringTok{ }\DecValTok{100}\NormalTok{, T)}
  \KeywordTok{return}\NormalTok{(rel_temp)}
\NormalTok{\}}
\end{Highlighting}
\end{Shaded}

Where T is the BAU temperature increase at year 100. For example, if T =
5 then the temperature increases over time (linearly) until year 100,
when it flattens out at 5.

The hotter it is, the more it affects daily life and it starts to eat
away at economic activity. Let K(t) be the fraction of economic activity
that is retained in a year if the temperature is tau (t), given by

K(t) = exp(−beta * tau(t)\^{}2) (2)

\begin{Shaded}
\begin{Highlighting}[]
\CommentTok{# Fraction of economic activity retained}
\NormalTok{K_t <-}\StringTok{ }\ControlFlowTok{function}\NormalTok{(beta, tau_t)\{}
\NormalTok{  K_t =}\StringTok{ }\KeywordTok{exp}\NormalTok{(}\OperatorTok{-}\NormalTok{beta }\OperatorTok{*}\StringTok{ }\NormalTok{tau_t}\OperatorTok{^}\DecValTok{2}\NormalTok{)}
  \KeywordTok{return}\NormalTok{(K_t)}
\NormalTok{\}}
\end{Highlighting}
\end{Shaded}

Economic activity (``consumption'') grows over time at rate g, but is
reduced by K (see above), so total consumption at time t is:

C(t) = K(t)exp(gt) (3)

\begin{Shaded}
\begin{Highlighting}[]
\CommentTok{# Consumption retained}
\NormalTok{C_t <-}\StringTok{ }\ControlFlowTok{function}\NormalTok{(K_t, g, t)\{}
\NormalTok{  C_t <-}\StringTok{ }\NormalTok{K_t }\OperatorTok{*}\StringTok{ }\KeywordTok{exp}\NormalTok{(g}\OperatorTok{*}\NormalTok{t)}
  \KeywordTok{return}\NormalTok{(C_t)}
\NormalTok{\}}
\end{Highlighting}
\end{Shaded}

Society's utility from consumption is given by the function

U(C) = ( C\^{}(1 − eta) ) / (1 - eta) (4)

\begin{Shaded}
\begin{Highlighting}[]
\CommentTok{# Utility from consumption}
\NormalTok{U_C <-}\StringTok{ }\ControlFlowTok{function}\NormalTok{(C, eta)\{}
\NormalTok{  U_C =}\StringTok{ }\NormalTok{(C }\OperatorTok{^}\NormalTok{(}\DecValTok{1} \OperatorTok{-}\StringTok{ }\NormalTok{eta)) }\OperatorTok{/}\StringTok{ }\NormalTok{(}\DecValTok{1} \OperatorTok{-}\StringTok{ }\NormalTok{eta)}
  \KeywordTok{return}\NormalTok{(U_C)}
\NormalTok{\}}
\end{Highlighting}
\end{Shaded}

For some analyses below, you may wish to discount utility to present
value. The discount rate is given by the Ramsey Rule:

r = delta + eta * g (5)

\begin{Shaded}
\begin{Highlighting}[]
\NormalTok{ramsey <-}\StringTok{ }\ControlFlowTok{function}\NormalTok{(delta, eta, g)\{}
\NormalTok{  r =}\StringTok{ }\NormalTok{delta }\OperatorTok{+}\StringTok{ }\NormalTok{eta }\OperatorTok{*}\StringTok{ }\NormalTok{g}
  \KeywordTok{return}\NormalTok{(r)}
\NormalTok{\}}
\end{Highlighting}
\end{Shaded}

You will build a climate-economy model using the equations above. Use
the following base case parameters for this model: delta = 0.005, eta =
0.5, g = 0.01, beta = 0.05.

\begin{Shaded}
\begin{Highlighting}[]
\NormalTok{delta <-}\StringTok{ }\FloatTok{0.005}
\NormalTok{eta <-}\StringTok{ }\FloatTok{0.5}
\NormalTok{g <-}\StringTok{ }\FloatTok{0.01}
\NormalTok{beta <-}\StringTok{ }\FloatTok{0.05}
\end{Highlighting}
\end{Shaded}

\begin{center}\rule{0.5\linewidth}{0.5pt}\end{center}

\begin{enumerate}
\def\labelenumi{\arabic{enumi}.}
\tightlist
\item
  Plots
\end{enumerate}

\begin{enumerate}
\def\labelenumi{(\alph{enumi})}
\tightlist
\item
  Plot temperature over time for no climate change (T = 0), with modest
  climate change (T = 2), and with extreme climate change (T = 8).
\end{enumerate}

\begin{Shaded}
\begin{Highlighting}[]
\NormalTok{df_plots <-}\StringTok{ }\NormalTok{df }\OperatorTok
\StringTok{  }\KeywordTok{mutate}\NormalTok{(}\DataTypeTok{tau_t_0 =} \KeywordTok{rel_temp_t}\NormalTok{(}\DecValTok{0}\NormalTok{, t),}
         \DataTypeTok{tau_t_2 =} \KeywordTok{rel_temp_t}\NormalTok{(}\DecValTok{2}\NormalTok{, t),}
         \DataTypeTok{tau_t_8 =} \KeywordTok{rel_temp_t}\NormalTok{(}\DecValTok{8}\NormalTok{, t))}

\CommentTok{# Plot: Temperature over time}

\KeywordTok{ggplot}\NormalTok{(}\DataTypeTok{data =}\NormalTok{ df_plots) }\OperatorTok{+}
\StringTok{  }\KeywordTok{geom_line}\NormalTok{(}\KeywordTok{aes}\NormalTok{(}\DataTypeTok{x =}\NormalTok{ t,}
                \DataTypeTok{y =}\NormalTok{ tau_t_}\DecValTok{0}\NormalTok{),}
            \DataTypeTok{color =} \StringTok{"blue"}\NormalTok{) }\OperatorTok{+}
\StringTok{  }\KeywordTok{geom_line}\NormalTok{(}\KeywordTok{aes}\NormalTok{(}\DataTypeTok{x =}\NormalTok{ t,}
                \DataTypeTok{y =}\NormalTok{ tau_t_}\DecValTok{2}\NormalTok{),}
            \DataTypeTok{color =} \StringTok{"orangered"}\NormalTok{) }\OperatorTok{+}
\StringTok{  }\KeywordTok{geom_line}\NormalTok{(}\KeywordTok{aes}\NormalTok{(}\DataTypeTok{x =}\NormalTok{ t,}
                \DataTypeTok{y =}\NormalTok{ tau_t_}\DecValTok{8}\NormalTok{),}
            \DataTypeTok{color =} \StringTok{"dark red"}\NormalTok{) }\OperatorTok{+}
\StringTok{  }\KeywordTok{labs}\NormalTok{(}\DataTypeTok{title =} \StringTok{"Relative Temperature"}\NormalTok{,}
       \CommentTok{# subtitle = "",}
       \DataTypeTok{x =} \StringTok{"Year"}\NormalTok{,}
       \DataTypeTok{y =} \StringTok{"Relative Temperature"}\NormalTok{,}
       \DataTypeTok{caption =} \StringTok{"Figure 1. Relative temperature change over time for various climate change scenarios.}\CharTok{\textbackslash{}n}\StringTok{ESM 204 Spring 2020}\CharTok{\textbackslash{}n}\StringTok{Bobby Miyashiro & Keene Morrow"}\NormalTok{) }\OperatorTok{+}
\StringTok{  }\CommentTok{# scale_x_continuous(expand = c(0, 0)) +}
\StringTok{  }\KeywordTok{theme_minimal}\NormalTok{() }\OperatorTok{+}
\StringTok{  }\KeywordTok{theme}\NormalTok{(}\DataTypeTok{plot.caption =} \KeywordTok{element_text}\NormalTok{(}\DataTypeTok{hjust =} \DecValTok{0}\NormalTok{, }\DataTypeTok{face =} \StringTok{"italic"}\NormalTok{))}
\end{Highlighting}
\end{Shaded}

\includegraphics{esm204_HW4_miyashiro_morrow_files/figure-latex/unnamed-chunk-7-1.pdf}

\begin{enumerate}
\def\labelenumi{(\alph{enumi})}
\setcounter{enumi}{1}
\tightlist
\item
  Plot consumption over time for no climate change, modest climate
  change, and extreme climate change.
\end{enumerate}

\begin{Shaded}
\begin{Highlighting}[]
\NormalTok{df_plots <-}\StringTok{ }\NormalTok{df_plots }\OperatorTok
\StringTok{  }\KeywordTok{mutate}\NormalTok{(}\DataTypeTok{K_0 =} \KeywordTok{K_t}\NormalTok{(beta, tau_t_}\DecValTok{0}\NormalTok{),}
         \DataTypeTok{K_2 =} \KeywordTok{K_t}\NormalTok{(beta, tau_t_}\DecValTok{2}\NormalTok{),}
         \DataTypeTok{K_8 =} \KeywordTok{K_t}\NormalTok{(beta, tau_t_}\DecValTok{8}\NormalTok{),}
         \DataTypeTok{C_0 =} \KeywordTok{C_t}\NormalTok{(K_}\DecValTok{0}\NormalTok{, g, t),}
         \DataTypeTok{C_2 =} \KeywordTok{C_t}\NormalTok{(K_}\DecValTok{2}\NormalTok{, g, t),}
         \DataTypeTok{C_8 =} \KeywordTok{C_t}\NormalTok{(K_}\DecValTok{8}\NormalTok{, g, t))}

\CommentTok{# Plot: Consumption}
\KeywordTok{ggplot}\NormalTok{(}\DataTypeTok{data =}\NormalTok{ df_plots) }\OperatorTok{+}
\StringTok{  }\KeywordTok{geom_line}\NormalTok{(}\KeywordTok{aes}\NormalTok{(}\DataTypeTok{x =}\NormalTok{ t,}
                \DataTypeTok{y =}\NormalTok{ C_}\DecValTok{0}\NormalTok{),}
            \DataTypeTok{color =} \StringTok{"blue"}\NormalTok{) }\OperatorTok{+}
\StringTok{  }\KeywordTok{geom_line}\NormalTok{(}\KeywordTok{aes}\NormalTok{(}\DataTypeTok{x =}\NormalTok{ t,}
                \DataTypeTok{y =}\NormalTok{ C_}\DecValTok{2}\NormalTok{),}
            \DataTypeTok{color =} \StringTok{"orangered"}\NormalTok{) }\OperatorTok{+}
\StringTok{  }\KeywordTok{geom_line}\NormalTok{(}\KeywordTok{aes}\NormalTok{(}\DataTypeTok{x =}\NormalTok{ t,}
                \DataTypeTok{y =}\NormalTok{ C_}\DecValTok{8}\NormalTok{),}
            \DataTypeTok{color =} \StringTok{"dark red"}\NormalTok{) }\OperatorTok{+}
\StringTok{  }\KeywordTok{labs}\NormalTok{(}\DataTypeTok{title =} \StringTok{"Consumption"}\NormalTok{,}
       \CommentTok{# subtitle = "",}
       \DataTypeTok{x =} \StringTok{"Year"}\NormalTok{,}
       \DataTypeTok{y =} \StringTok{"Retained Consumption"}\NormalTok{,}
       \DataTypeTok{caption =} \StringTok{"Figure 2. Retained consumption over time for various climate change scenarios.}\CharTok{\textbackslash{}n}\StringTok{Temperature change in year 100: blue = 0, orange = 2, red = 8}\CharTok{\textbackslash{}n\textbackslash{}n}\StringTok{ESM 204 Spring 2020}\CharTok{\textbackslash{}n}\StringTok{Bobby Miyashiro & Keene Morrow"}\NormalTok{) }\OperatorTok{+}
\StringTok{  }\CommentTok{# scale_x_continuous(expand = c(0, 0)) +}
\StringTok{  }\KeywordTok{theme_minimal}\NormalTok{() }\OperatorTok{+}
\StringTok{  }\KeywordTok{theme}\NormalTok{(}\DataTypeTok{plot.caption =} \KeywordTok{element_text}\NormalTok{(}\DataTypeTok{hjust =} \DecValTok{0}\NormalTok{, }\DataTypeTok{face =} \StringTok{"italic"}\NormalTok{))}
\end{Highlighting}
\end{Shaded}

\includegraphics{esm204_HW4_miyashiro_morrow_files/figure-latex/unnamed-chunk-8-1.pdf}

\begin{enumerate}
\def\labelenumi{(\alph{enumi})}
\setcounter{enumi}{2}
\tightlist
\item
  Plot undiscounted utility over time for no climate change, modest
  climate change, and extreme climate change.
\end{enumerate}

\begin{Shaded}
\begin{Highlighting}[]
\NormalTok{df_plots <-}\StringTok{ }\NormalTok{df_plots }\OperatorTok
\StringTok{  }\KeywordTok{mutate}\NormalTok{(}\DataTypeTok{U_0 =} \KeywordTok{U_C}\NormalTok{(C_}\DecValTok{0}\NormalTok{, eta),}
         \DataTypeTok{U_2 =} \KeywordTok{U_C}\NormalTok{(C_}\DecValTok{2}\NormalTok{, eta),}
         \DataTypeTok{U_8 =} \KeywordTok{U_C}\NormalTok{(C_}\DecValTok{8}\NormalTok{, eta))}

\CommentTok{# Plot: undiscounted utility}
\KeywordTok{ggplot}\NormalTok{(}\DataTypeTok{data =}\NormalTok{ df_plots) }\OperatorTok{+}
\StringTok{  }\KeywordTok{geom_line}\NormalTok{(}\KeywordTok{aes}\NormalTok{(}\DataTypeTok{x =}\NormalTok{ t,}
                \DataTypeTok{y =}\NormalTok{ U_}\DecValTok{0}\NormalTok{),}
            \DataTypeTok{color =} \StringTok{"blue"}\NormalTok{) }\OperatorTok{+}
\StringTok{  }\KeywordTok{geom_line}\NormalTok{(}\KeywordTok{aes}\NormalTok{(}\DataTypeTok{x =}\NormalTok{ t,}
                \DataTypeTok{y =}\NormalTok{ U_}\DecValTok{2}\NormalTok{),}
            \DataTypeTok{color =} \StringTok{"orangered"}\NormalTok{) }\OperatorTok{+}
\StringTok{  }\KeywordTok{geom_line}\NormalTok{(}\KeywordTok{aes}\NormalTok{(}\DataTypeTok{x =}\NormalTok{ t,}
                \DataTypeTok{y =}\NormalTok{ U_}\DecValTok{8}\NormalTok{),}
            \DataTypeTok{color =} \StringTok{"dark red"}\NormalTok{) }\OperatorTok{+}
\StringTok{  }\KeywordTok{labs}\NormalTok{(}\CommentTok{#title = "",}
       \CommentTok{# subtitle = "",}
       \DataTypeTok{x =} \StringTok{"Year"}\NormalTok{,}
       \DataTypeTok{y =} \StringTok{"Utility (Undiscounted)"}\NormalTok{,}
       \DataTypeTok{caption =} \StringTok{"Figure 3. Undiscounted utility over time for various climate change scenarios.}\CharTok{\textbackslash{}n}\StringTok{Temperature change in year 100: blue = 0, orange = 2, red = 8}\CharTok{\textbackslash{}n\textbackslash{}n}\StringTok{ESM 204 Spring 2020}\CharTok{\textbackslash{}n}\StringTok{Bobby Miyashiro & Keene Morrow"}\NormalTok{) }\OperatorTok{+}
\StringTok{  }\CommentTok{# scale_x_continuous(expand = c(0, 0)) +}
\StringTok{  }\KeywordTok{theme_minimal}\NormalTok{() }\OperatorTok{+}
\StringTok{  }\KeywordTok{theme}\NormalTok{(}\DataTypeTok{plot.caption =} \KeywordTok{element_text}\NormalTok{(}\DataTypeTok{hjust =} \DecValTok{0}\NormalTok{, }\DataTypeTok{face =} \StringTok{"italic"}\NormalTok{))}
\end{Highlighting}
\end{Shaded}

\includegraphics{esm204_HW4_miyashiro_morrow_files/figure-latex/unnamed-chunk-9-1.pdf}

\begin{enumerate}
\def\labelenumi{\arabic{enumi}.}
\setcounter{enumi}{1}
\tightlist
\item
  Analysis
\end{enumerate}

\begin{enumerate}
\def\labelenumi{(\alph{enumi})}
\tightlist
\item
  Suppose T = 4.4. In other words, suppose we know for sure that under
  BAU, climate change will eventually lead to a 4.4 degree increase in
  temperature.
\end{enumerate}

\begin{Shaded}
\begin{Highlighting}[]
\NormalTok{PV <-}\StringTok{ }\ControlFlowTok{function}\NormalTok{(x, r, t)\{}
\NormalTok{  PV =}\StringTok{ }\NormalTok{x}\OperatorTok{/}\NormalTok{(}\DecValTok{1}\OperatorTok{+}\NormalTok{r)}\OperatorTok{^}\NormalTok{t}
  \KeywordTok{return}\NormalTok{(PV)}
\NormalTok{\}}
\end{Highlighting}
\end{Shaded}

What is the present value (i.e.~discounted) utility over the next 200
years with climate change?

\begin{Shaded}
\begin{Highlighting}[]
\CommentTok{#Present value utility with 4.4 degree increase in temp from climate change}
\NormalTok{BAU_}\FloatTok{4.4}\NormalTok{ <-}\StringTok{ }\NormalTok{df }\OperatorTok\StringTok{ }
\StringTok{  }\KeywordTok{mutate}\NormalTok{(}\DataTypeTok{tau_t_4.4 =} \KeywordTok{rel_temp_t}\NormalTok{(}\FloatTok{4.4}\NormalTok{, t),}
         \DataTypeTok{K_4.4 =} \KeywordTok{K_t}\NormalTok{(beta, tau_t_}\FloatTok{4.4}\NormalTok{),}
         \DataTypeTok{C_4.4 =} \KeywordTok{C_t}\NormalTok{(K_}\FloatTok{4.4}\NormalTok{, g, t),}
         \DataTypeTok{U_4.4 =} \KeywordTok{U_C}\NormalTok{(C_}\FloatTok{4.4}\NormalTok{, eta),}
         \DataTypeTok{PV_4.4 =} \KeywordTok{PV}\NormalTok{(U_}\FloatTok{4.4}\NormalTok{, }\KeywordTok{ramsey}\NormalTok{(delta, eta, g), t))}

\NormalTok{PV_BAU_}\FloatTok{4.4}\NormalTok{ <-}\StringTok{ }\KeywordTok{sum}\NormalTok{(BAU_}\FloatTok{4.4}\OperatorTok{$}\NormalTok{PV_}\FloatTok{4.4}\NormalTok{)}
\CommentTok{#PV_BAU_4.4 is 198.661}
\end{Highlighting}
\end{Shaded}

The present value utility over the next 200 years is found by plugging a
T of 4.4 into the above equations and discounting over 200 years
resulting in a PV of \$198.66.

What is the present value utility without climate change?

\begin{Shaded}
\begin{Highlighting}[]
\CommentTok{#Present value of utility without climate, hence 0 degree increase}
\NormalTok{BAU_no_cc <-}\StringTok{ }\NormalTok{df }\OperatorTok\StringTok{ }
\StringTok{  }\KeywordTok{mutate}\NormalTok{(}\DataTypeTok{tau_t_no_cc =} \KeywordTok{rel_temp_t}\NormalTok{(}\DecValTok{0}\NormalTok{, t),}
         \DataTypeTok{K_no_cc =} \KeywordTok{K_t}\NormalTok{(beta, tau_t_no_cc),}
         \DataTypeTok{C_no_cc =} \KeywordTok{C_t}\NormalTok{(K_no_cc, g, t),}
         \DataTypeTok{U_no_cc =} \KeywordTok{U_C}\NormalTok{(C_no_cc, eta),}
         \DataTypeTok{PV_no_cc =} \KeywordTok{PV}\NormalTok{(U_no_cc, }\KeywordTok{ramsey}\NormalTok{(delta, eta, g), t))}

\NormalTok{PV_no_cc <-}\StringTok{ }\KeywordTok{sum}\NormalTok{(BAU_no_cc}\OperatorTok{$}\NormalTok{PV_no_cc)}
\end{Highlighting}
\end{Shaded}

The present value utilty over the next 200 years without climate change
is found by plugging a T of 0 into the above equations and discounting
over 200 years resulting in a PV of \$255.27.

What is the percentage loss in present value utility from climate change
(call this L)?

\begin{Shaded}
\begin{Highlighting}[]
\CommentTok{# Percent loss }
\NormalTok{L_}\FloatTok{4.4}\NormalTok{ <-}\StringTok{ }\NormalTok{((PV_no_cc }\OperatorTok{-}\StringTok{ }\NormalTok{PV_BAU_}\FloatTok{4.4}\NormalTok{)}\OperatorTok{/}\NormalTok{PV_no_cc)}\OperatorTok{*}\DecValTok{100}
\end{Highlighting}
\end{Shaded}

The percentage loss, in this case represented as L, is the difference
between PV with no climate change and PV with a temperature increase of
4.4, all divided by a PV with no climate change to result in 22.18\%
loss.

\begin{enumerate}
\def\labelenumi{(\alph{enumi})}
\setcounter{enumi}{1}
\tightlist
\item
  Now show how sensitive your calculation of L is to the following
  parameters: T, g, eta, and beta. To do so, calculate the \% change in
  L that arises from a 10\% increase in each of these parameters.
\end{enumerate}

\begin{Shaded}
\begin{Highlighting}[]
\CommentTok{# Present value utility with 4.84 degree increase in temp from climate change}
\NormalTok{BAU_}\FloatTok{4.84}\NormalTok{ <-}\StringTok{ }\NormalTok{df }\OperatorTok\StringTok{ }
\StringTok{  }\KeywordTok{mutate}\NormalTok{(}\DataTypeTok{tau_t_4.84 =} \KeywordTok{rel_temp_t}\NormalTok{(}\FloatTok{4.84}\NormalTok{, t),}
         \DataTypeTok{K_4.84 =} \KeywordTok{K_t}\NormalTok{(beta, tau_t_}\FloatTok{4.84}\NormalTok{),}
         \DataTypeTok{C_4.84 =} \KeywordTok{C_t}\NormalTok{(K_}\FloatTok{4.84}\NormalTok{, g, t),}
         \DataTypeTok{U_4.84 =} \KeywordTok{U_C}\NormalTok{(C_}\FloatTok{4.84}\NormalTok{, eta),}
         \DataTypeTok{PV_4.84 =} \KeywordTok{PV}\NormalTok{(U_}\FloatTok{4.84}\NormalTok{, }\KeywordTok{ramsey}\NormalTok{(delta, eta, g), t))}

\NormalTok{PV_BAU_}\FloatTok{4.84}\NormalTok{ <-}\StringTok{ }\KeywordTok{sum}\NormalTok{(BAU_}\FloatTok{4.84}\OperatorTok{$}\NormalTok{PV_}\FloatTok{4.84}\NormalTok{)}

\CommentTok{# Percent loss }
\NormalTok{L_}\FloatTok{4.84}\NormalTok{ <-}\StringTok{ }\NormalTok{((PV_no_cc }\OperatorTok{-}\StringTok{ }\NormalTok{PV_BAU_}\FloatTok{4.84}\NormalTok{)}\OperatorTok{/}\NormalTok{PV_no_cc)}\OperatorTok{*}\DecValTok{100}
\end{Highlighting}
\end{Shaded}

\begin{Shaded}
\begin{Highlighting}[]
\NormalTok{sen_test <-}\StringTok{ }\ControlFlowTok{function}\NormalTok{(Temp, g, eta, beta, }\DataTypeTok{delta =} \FloatTok{0.005}\NormalTok{)\{}
\NormalTok{  df <-}\StringTok{ }\KeywordTok{data.frame}\NormalTok{(}\DataTypeTok{t =} \KeywordTok{seq}\NormalTok{(}\DecValTok{0}\NormalTok{, }\DecValTok{200}\NormalTok{, }\DataTypeTok{by =} \DecValTok{1}\NormalTok{))}
  
\NormalTok{  T_no_cc <-}\StringTok{ }\DecValTok{0}
\NormalTok{  g_no_cc <-}\StringTok{ }\FloatTok{0.01}
\NormalTok{  eta_no_cc <-}\StringTok{ }\FloatTok{0.5}
\NormalTok{  beta_no_cc <-}\StringTok{ }\FloatTok{0.05}
\NormalTok{  delta_no_cc <-}\StringTok{ }\FloatTok{0.005}
  
\NormalTok{  BAU_no_cc <-}\StringTok{ }\NormalTok{df }\OperatorTok\StringTok{ }
\StringTok{    }\KeywordTok{mutate}\NormalTok{(}\DataTypeTok{tau_t_no_cc =} \KeywordTok{rel_temp_t}\NormalTok{(T_no_cc, t),}
           \DataTypeTok{K_no_cc =} \KeywordTok{K_t}\NormalTok{(beta_no_cc, tau_t_no_cc),}
           \DataTypeTok{C_no_cc =} \KeywordTok{C_t}\NormalTok{(K_no_cc, g_no_cc, t),}
           \DataTypeTok{U_no_cc =} \KeywordTok{U_C}\NormalTok{(C_no_cc, eta_no_cc),}
           \DataTypeTok{PV_no_cc =} \KeywordTok{PV}\NormalTok{(U_no_cc, }\KeywordTok{ramsey}\NormalTok{(delta_no_cc, eta_no_cc, g_no_cc), t))}
  
\NormalTok{  PV_no_cc <-}\StringTok{ }\KeywordTok{sum}\NormalTok{(BAU_no_cc}\OperatorTok{$}\NormalTok{PV_no_cc)}
  
\NormalTok{  BAU <-}\StringTok{ }\NormalTok{df }\OperatorTok\StringTok{ }
\StringTok{    }\KeywordTok{mutate}\NormalTok{(}\DataTypeTok{tau_t =} \KeywordTok{rel_temp_t}\NormalTok{(Temp, t),}
           \DataTypeTok{K =} \KeywordTok{K_t}\NormalTok{(beta, tau_t),}
           \DataTypeTok{C =} \KeywordTok{C_t}\NormalTok{(K, g, t),}
           \DataTypeTok{U =} \KeywordTok{U_C}\NormalTok{(C, eta),}
           \DataTypeTok{PV =} \KeywordTok{PV}\NormalTok{(U, }\KeywordTok{ramsey}\NormalTok{(delta, eta, g), t))}
  
\NormalTok{  PV_BAU <-}\StringTok{ }\KeywordTok{sum}\NormalTok{(BAU}\OperatorTok{$}\NormalTok{PV)}
  
\NormalTok{  L =}\StringTok{ }\NormalTok{((PV_no_cc }\OperatorTok{-}\StringTok{ }\NormalTok{PV_BAU)}\OperatorTok{/}\NormalTok{PV_no_cc)}\OperatorTok{*}\DecValTok{100}
  
  \KeywordTok{return}\NormalTok{(L)}
\NormalTok{\}}

\NormalTok{df_2b <-}\StringTok{ }\KeywordTok{data.frame}\NormalTok{(}\DataTypeTok{increase =} \KeywordTok{c}\NormalTok{(}\StringTok{"none"}\NormalTok{, }\StringTok{"T"}\NormalTok{, }\StringTok{"g"}\NormalTok{, }\StringTok{"eta"}\NormalTok{, }\StringTok{"beta"}\NormalTok{)) }\OperatorTok
\StringTok{  }\KeywordTok{mutate}\NormalTok{(}\DataTypeTok{T_sen =} \KeywordTok{as.numeric}\NormalTok{(}\KeywordTok{c}\NormalTok{(}\FloatTok{4.4}\NormalTok{, }\FloatTok{4.84}\NormalTok{, }\FloatTok{4.4}\NormalTok{, }\FloatTok{4.4}\NormalTok{, }\FloatTok{4.4}\NormalTok{)),}
         \DataTypeTok{g_sen =} \KeywordTok{as.numeric}\NormalTok{(}\KeywordTok{c}\NormalTok{(g, g, g }\OperatorTok{*}\StringTok{ }\FloatTok{1.1}\NormalTok{, g, g)),}
         \DataTypeTok{eta_sen =} \KeywordTok{as.numeric}\NormalTok{(}\KeywordTok{c}\NormalTok{(eta, eta, eta, eta }\OperatorTok{*}\StringTok{ }\FloatTok{1.1}\NormalTok{, eta)),}
         \DataTypeTok{beta_sen =} \KeywordTok{as.numeric}\NormalTok{(}\KeywordTok{c}\NormalTok{(beta, beta, beta, beta, beta }\OperatorTok{*}\StringTok{ }\FloatTok{1.1}\NormalTok{)))}



\NormalTok{df_2b <-}\StringTok{ }\NormalTok{df_2b }\OperatorTok\StringTok{ }
\StringTok{  }\KeywordTok{mutate}\NormalTok{(}\DataTypeTok{L_KM =} \KeywordTok{c}\NormalTok{(}\KeywordTok{sen_test}\NormalTok{(df_2b}\OperatorTok{$}\NormalTok{T_sen[}\DecValTok{1}\NormalTok{], df_2b}\OperatorTok{$}\NormalTok{g_sen[}\DecValTok{1}\NormalTok{], df_2b}\OperatorTok{$}\NormalTok{eta_sen[}\DecValTok{1}\NormalTok{], df_2b}\OperatorTok{$}\NormalTok{beta_sen[}\DecValTok{1}\NormalTok{], delta),}
                  \KeywordTok{sen_test}\NormalTok{(df_2b}\OperatorTok{$}\NormalTok{T_sen[}\DecValTok{2}\NormalTok{], df_2b}\OperatorTok{$}\NormalTok{g_sen[}\DecValTok{2}\NormalTok{], df_2b}\OperatorTok{$}\NormalTok{eta_sen[}\DecValTok{2}\NormalTok{], df_2b}\OperatorTok{$}\NormalTok{beta_sen[}\DecValTok{2}\NormalTok{], delta),}
                  \KeywordTok{sen_test}\NormalTok{(df_2b}\OperatorTok{$}\NormalTok{T_sen[}\DecValTok{3}\NormalTok{], df_2b}\OperatorTok{$}\NormalTok{g_sen[}\DecValTok{3}\NormalTok{], df_2b}\OperatorTok{$}\NormalTok{eta_sen[}\DecValTok{3}\NormalTok{], df_2b}\OperatorTok{$}\NormalTok{beta_sen[}\DecValTok{3}\NormalTok{], delta),}
                  \KeywordTok{sen_test}\NormalTok{(df_2b}\OperatorTok{$}\NormalTok{T_sen[}\DecValTok{4}\NormalTok{], df_2b}\OperatorTok{$}\NormalTok{g_sen[}\DecValTok{4}\NormalTok{], df_2b}\OperatorTok{$}\NormalTok{eta_sen[}\DecValTok{4}\NormalTok{], df_2b}\OperatorTok{$}\NormalTok{beta_sen[}\DecValTok{4}\NormalTok{], delta),}
                  \KeywordTok{sen_test}\NormalTok{(df_2b}\OperatorTok{$}\NormalTok{T_sen[}\DecValTok{5}\NormalTok{], df_2b}\OperatorTok{$}\NormalTok{g_sen[}\DecValTok{5}\NormalTok{], df_2b}\OperatorTok{$}\NormalTok{eta_sen[}\DecValTok{5}\NormalTok{], df_2b}\OperatorTok{$}\NormalTok{beta_sen[}\DecValTok{5}\NormalTok{], delta))) }\OperatorTok
\StringTok{  }\KeywordTok{mutate}\NormalTok{(}\DataTypeTok{L_BM =} \KeywordTok{c}\NormalTok{(}\OtherTok{NA}\NormalTok{, }\FloatTok{25.797}\NormalTok{, }\FloatTok{22.181}\NormalTok{, }\FloatTok{19.596}\NormalTok{, }\FloatTok{23.939}\NormalTok{),}
         \DataTypeTok{L_diff =}\NormalTok{ L_KM }\OperatorTok{-}\StringTok{ }\NormalTok{L_BM)}

\NormalTok{L_no_cc <-}\StringTok{ }\NormalTok{df_2b}\OperatorTok{$}\NormalTok{L_KM[}\DecValTok{1}\NormalTok{]}

\NormalTok{df_2b <-}\StringTok{ }\NormalTok{df_2b }\OperatorTok\StringTok{ }
\StringTok{  }\KeywordTok{rowwise}\NormalTok{() }\OperatorTok\StringTok{ }
\StringTok{  }\KeywordTok{mutate}\NormalTok{(}\DataTypeTok{L_pct =}\NormalTok{ ((L_KM }\OperatorTok{-}\StringTok{ }\NormalTok{L_no_cc)}\OperatorTok{/}\NormalTok{L_no_cc)}\OperatorTok{*}\DecValTok{100}\NormalTok{)}
\end{Highlighting}
\end{Shaded}

\begin{table}[H]
\centering
\begin{tabular}{l|r}
\hline
\multicolumn{2}{c|}{Sensitivity Test: Present Value Utility Lost (L)} \\
\cline{1-2}
Adjusted Value & Percent Change in L\\
\hline
T & 16.3217768\\
\hline
g & -0.1301919\\
\hline
eta & -20.3563263\\
\hline
beta & 7.9429826\\
\hline
\end{tabular}
\end{table}

The table above shows the percent change in L for a 10\% increase in T,
g, eta, and beta.

\begin{enumerate}
\def\labelenumi{(\alph{enumi})}
\setcounter{enumi}{2}
\tightlist
\item
  Back to the original parameters, suppose we could completely prevent
  climate change from occurring (so T = 0 instead of T = 4.4) but doing
  so would require giving up a fraction theta of consumption every year
  for the next 200 years. What is the maximum value of theta society
  would be willing to endure every year to completely prevent climate
  change? Call this theta*.
\end{enumerate}

\begin{Shaded}
\begin{Highlighting}[]
\NormalTok{C_t_theta <-}\StringTok{ }\ControlFlowTok{function}\NormalTok{(K_t, g, t, }\DataTypeTok{theta =} \DecValTok{0}\NormalTok{)\{}
\NormalTok{  C_t <-}\StringTok{ }\NormalTok{(K_t }\OperatorTok{*}\StringTok{ }\KeywordTok{exp}\NormalTok{(g}\OperatorTok{*}\NormalTok{t)) }\OperatorTok{*}\StringTok{ }\NormalTok{(}\DecValTok{1} \OperatorTok{-}\StringTok{ }\NormalTok{theta)}
  \KeywordTok{return}\NormalTok{(C_t)}
\NormalTok{\}}

\NormalTok{T_no_cc <-}\StringTok{ }\DecValTok{0}
\NormalTok{g_no_cc <-}\StringTok{ }\FloatTok{0.01}
\NormalTok{eta_no_cc <-}\StringTok{ }\FloatTok{0.5}
\NormalTok{beta_no_cc <-}\StringTok{ }\FloatTok{0.05}
\NormalTok{delta_no_cc <-}\StringTok{ }\FloatTok{0.005}
\NormalTok{theta <-}\StringTok{ }\DecValTok{0}

\NormalTok{BAU_no_cc <-}\StringTok{ }\NormalTok{df }\OperatorTok\StringTok{ }
\StringTok{  }\KeywordTok{mutate}\NormalTok{(}\DataTypeTok{tau_t_no_cc =} \KeywordTok{rel_temp_t}\NormalTok{(T_no_cc, t),}
         \DataTypeTok{K_no_cc =} \KeywordTok{K_t}\NormalTok{(beta_no_cc, tau_t_no_cc),}
         \DataTypeTok{C_no_cc =} \KeywordTok{C_t_theta}\NormalTok{(K_no_cc, g_no_cc, t, theta),}
         \DataTypeTok{U_no_cc =} \KeywordTok{U_C}\NormalTok{(C_no_cc, eta_no_cc),}
         \DataTypeTok{PV_no_cc =} \KeywordTok{PV}\NormalTok{(U_no_cc, }\KeywordTok{ramsey}\NormalTok{(delta_no_cc, eta_no_cc, g_no_cc), t))}
  
\NormalTok{PV_BAU_no_cc <-}\StringTok{ }\KeywordTok{sum}\NormalTok{(BAU_no_cc}\OperatorTok{$}\NormalTok{PV)}

\NormalTok{BAU_}\FloatTok{4.4}\NormalTok{ <-}\StringTok{ }\NormalTok{df }\OperatorTok\StringTok{ }
\StringTok{  }\KeywordTok{mutate}\NormalTok{(}\DataTypeTok{tau_t_no_cc =} \KeywordTok{rel_temp_t}\NormalTok{(}\FloatTok{4.4}\NormalTok{, t),}
         \DataTypeTok{K_no_cc =} \KeywordTok{K_t}\NormalTok{(beta_no_cc, tau_t_no_cc),}
         \DataTypeTok{C_no_cc =} \KeywordTok{C_t_theta}\NormalTok{(K_no_cc, g_no_cc, t, theta),}
         \DataTypeTok{U_no_cc =} \KeywordTok{U_C}\NormalTok{(C_no_cc, eta_no_cc),}
         \DataTypeTok{PV_no_cc =} \KeywordTok{PV}\NormalTok{(U_no_cc, }\KeywordTok{ramsey}\NormalTok{(delta_no_cc, eta_no_cc, g_no_cc), t))}

\NormalTok{PV_BAU_}\FloatTok{4.4}\NormalTok{ <-}\StringTok{ }\KeywordTok{sum}\NormalTok{(BAU_}\FloatTok{4.4}\OperatorTok{$}\NormalTok{PV)}

\NormalTok{theta_vector <-}\StringTok{ }\KeywordTok{seq}\NormalTok{(}\FloatTok{0.3943590}\NormalTok{, }\FloatTok{0.3943600}\NormalTok{, }\DataTypeTok{by =} \FloatTok{0.00000001}\NormalTok{)}

\NormalTok{df_2c <-}\StringTok{ }\KeywordTok{data.frame}\NormalTok{(}\DataTypeTok{theta =} \DecValTok{0}\NormalTok{,}
                    \DataTypeTok{PV =} \DecValTok{0}\NormalTok{)}

\ControlFlowTok{for}\NormalTok{(i }\ControlFlowTok{in}\NormalTok{ theta_vector)\{}
\NormalTok{  BAU_reduce <-}\StringTok{ }\NormalTok{df }\OperatorTok\StringTok{ }
\StringTok{  }\KeywordTok{mutate}\NormalTok{(}\DataTypeTok{tau_t_no_cc =} \KeywordTok{rel_temp_t}\NormalTok{(T_no_cc, t),}
         \DataTypeTok{K_no_cc =} \KeywordTok{K_t}\NormalTok{(beta_no_cc, tau_t_no_cc),}
         \DataTypeTok{C_no_cc =} \KeywordTok{C_t_theta}\NormalTok{(K_no_cc, g_no_cc, t, i),}
         \DataTypeTok{U_no_cc =} \KeywordTok{U_C}\NormalTok{(C_no_cc, eta_no_cc),}
         \DataTypeTok{PV_no_cc =} \KeywordTok{PV}\NormalTok{(U_no_cc, }\KeywordTok{ramsey}\NormalTok{(delta_no_cc, eta_no_cc, g_no_cc), t))}
  
\NormalTok{PV_BAU_reduce <-}\StringTok{ }\KeywordTok{sum}\NormalTok{(BAU_reduce}\OperatorTok{$}\NormalTok{PV)}

\NormalTok{df_temp <-}\StringTok{ }\KeywordTok{data.frame}\NormalTok{(}\DataTypeTok{theta =}\NormalTok{ i,}
                    \DataTypeTok{PV =} \KeywordTok{sum}\NormalTok{(BAU_reduce}\OperatorTok{$}\NormalTok{PV))}

\NormalTok{df_2c <-}\StringTok{ }\NormalTok{df_2c }\OperatorTok
\StringTok{  }\KeywordTok{full_join}\NormalTok{(df_temp)}
\NormalTok{\}}

\NormalTok{df_2c_sol <-}\StringTok{ }\NormalTok{df_2c }\OperatorTok
\StringTok{  }\KeywordTok{filter}\NormalTok{(}\KeywordTok{round}\NormalTok{(PV, }\DecValTok{4}\NormalTok{) }\OperatorTok{==}\StringTok{ }\FloatTok{198.6612}\NormalTok{)}

\NormalTok{theta_star <-}\StringTok{ }\KeywordTok{mean}\NormalTok{(df_2c_sol}\OperatorTok{$}\NormalTok{theta)}
\end{Highlighting}
\end{Shaded}

The maximum value of theta society would be willing to endure every year
to completely prevent climate change is theta* = 0.3944.

\begin{enumerate}
\def\labelenumi{(\alph{enumi})}
\setcounter{enumi}{3}
\tightlist
\item
  Suppose we are uncertain about T, but it has the following probability
  distribution: T = 2 (with probability .2), T = 4 (with probability
  .5), and T = 6 (with probability .3). Calculate theta* under
  uncertainty over T.
\end{enumerate}

\begin{Shaded}
\begin{Highlighting}[]
\NormalTok{df_2d <-}\StringTok{ }\KeywordTok{data.frame}\NormalTok{(}\DataTypeTok{T =} \KeywordTok{c}\NormalTok{(}\DecValTok{2}\NormalTok{, }\DecValTok{4}\NormalTok{, }\DecValTok{6}\NormalTok{),}
                    \DataTypeTok{p =} \KeywordTok{c}\NormalTok{(}\FloatTok{0.2}\NormalTok{, }\FloatTok{0.5}\NormalTok{, }\FloatTok{0.3}\NormalTok{))}

\CommentTok{# Present value utility with 2 degree increase in temp from climate change}
\NormalTok{BAU_}\DecValTok{2}\NormalTok{ <-}\StringTok{ }\NormalTok{df }\OperatorTok\StringTok{ }
\StringTok{  }\KeywordTok{mutate}\NormalTok{(}\DataTypeTok{tau_t_2 =} \KeywordTok{rel_temp_t}\NormalTok{(}\DecValTok{2}\NormalTok{, t),}
         \DataTypeTok{K_2 =} \KeywordTok{K_t}\NormalTok{(beta, tau_t_}\DecValTok{2}\NormalTok{),}
         \DataTypeTok{C_2 =} \KeywordTok{C_t}\NormalTok{(K_}\DecValTok{2}\NormalTok{, g, t),}
         \DataTypeTok{U_2 =} \KeywordTok{U_C}\NormalTok{(C_}\DecValTok{2}\NormalTok{, eta),}
         \DataTypeTok{PV_2 =} \KeywordTok{PV}\NormalTok{(U_}\DecValTok{2}\NormalTok{, }\KeywordTok{ramsey}\NormalTok{(delta, eta, g), t))}

\NormalTok{PV_BAU_}\DecValTok{2}\NormalTok{ <-}\StringTok{ }\KeywordTok{sum}\NormalTok{(BAU_}\DecValTok{2}\OperatorTok{$}\NormalTok{PV_}\DecValTok{2}\NormalTok{)}
\CommentTok{# PV_BAU_2}

\CommentTok{# Present value utility with 4 degree increase in temp from climate change}
\NormalTok{BAU_}\DecValTok{4}\NormalTok{ <-}\StringTok{ }\NormalTok{df }\OperatorTok\StringTok{ }
\StringTok{  }\KeywordTok{mutate}\NormalTok{(}\DataTypeTok{tau_t_4 =} \KeywordTok{rel_temp_t}\NormalTok{(}\DecValTok{4}\NormalTok{, t),}
         \DataTypeTok{K_4 =} \KeywordTok{K_t}\NormalTok{(beta, tau_t_}\DecValTok{4}\NormalTok{),}
         \DataTypeTok{C_4 =} \KeywordTok{C_t}\NormalTok{(K_}\DecValTok{4}\NormalTok{, g, t),}
         \DataTypeTok{U_4 =} \KeywordTok{U_C}\NormalTok{(C_}\DecValTok{4}\NormalTok{, eta),}
         \DataTypeTok{PV_4 =} \KeywordTok{PV}\NormalTok{(U_}\DecValTok{4}\NormalTok{, }\KeywordTok{ramsey}\NormalTok{(delta, eta, g), t))}

\NormalTok{PV_BAU_}\DecValTok{4}\NormalTok{ <-}\StringTok{ }\KeywordTok{sum}\NormalTok{(BAU_}\DecValTok{4}\OperatorTok{$}\NormalTok{PV_}\DecValTok{4}\NormalTok{)}
\CommentTok{# PV_BAU_4}

\CommentTok{# Present value utility with 6 degree increase in temp from climate change}
\NormalTok{BAU_}\DecValTok{6}\NormalTok{ <-}\StringTok{ }\NormalTok{df }\OperatorTok\StringTok{ }
\StringTok{  }\KeywordTok{mutate}\NormalTok{(}\DataTypeTok{tau_t_6 =} \KeywordTok{rel_temp_t}\NormalTok{(}\DecValTok{6}\NormalTok{, t),}
         \DataTypeTok{K_6 =} \KeywordTok{K_t}\NormalTok{(beta, tau_t_}\DecValTok{6}\NormalTok{),}
         \DataTypeTok{C_6 =} \KeywordTok{C_t}\NormalTok{(K_}\DecValTok{6}\NormalTok{, g, t),}
         \DataTypeTok{U_6 =} \KeywordTok{U_C}\NormalTok{(C_}\DecValTok{6}\NormalTok{, eta),}
         \DataTypeTok{PV_6 =} \KeywordTok{PV}\NormalTok{(U_}\DecValTok{6}\NormalTok{, }\KeywordTok{ramsey}\NormalTok{(delta, eta, g), t))}

\NormalTok{PV_BAU_}\DecValTok{6}\NormalTok{ <-}\StringTok{ }\KeywordTok{sum}\NormalTok{(BAU_}\DecValTok{6}\OperatorTok{$}\NormalTok{PV_}\DecValTok{6}\NormalTok{)}
\CommentTok{# PV_BAU_6}

\NormalTok{PV_uncertainty <-}\StringTok{ }\FloatTok{0.2}\OperatorTok{*}\NormalTok{PV_BAU_}\DecValTok{2} \OperatorTok{+}\StringTok{ }\FloatTok{0.5}\OperatorTok{*}\NormalTok{PV_BAU_}\DecValTok{4} \OperatorTok{+}\StringTok{ }\FloatTok{0.3}\OperatorTok{*}\NormalTok{PV_BAU_}\DecValTok{6}


\CommentTok{# Testing for theta}
\NormalTok{theta_vector_}\DecValTok{2}\NormalTok{ <-}\StringTok{ }\KeywordTok{seq}\NormalTok{(}\FloatTok{0.377886}\NormalTok{, }\FloatTok{0.377888}\NormalTok{, }\DataTypeTok{by =} \FloatTok{0.00000001}\NormalTok{)}

\NormalTok{df_2d <-}\StringTok{ }\KeywordTok{data.frame}\NormalTok{(}\DataTypeTok{theta =} \OtherTok{NA}\NormalTok{,}
                    \DataTypeTok{PV =} \OtherTok{NA}\NormalTok{)}

\ControlFlowTok{for}\NormalTok{(i }\ControlFlowTok{in}\NormalTok{ theta_vector_}\DecValTok{2}\NormalTok{)\{}
\NormalTok{  BAU_reduce <-}\StringTok{ }\NormalTok{df }\OperatorTok\StringTok{ }
\StringTok{  }\KeywordTok{mutate}\NormalTok{(}\DataTypeTok{tau_t_no_cc =} \KeywordTok{rel_temp_t}\NormalTok{(T_no_cc, t),}
         \DataTypeTok{K_no_cc =} \KeywordTok{K_t}\NormalTok{(beta_no_cc, tau_t_no_cc),}
         \DataTypeTok{C_no_cc =} \KeywordTok{C_t_theta}\NormalTok{(K_no_cc, g_no_cc, t, i),}
         \DataTypeTok{U_no_cc =} \KeywordTok{U_C}\NormalTok{(C_no_cc, eta_no_cc),}
         \DataTypeTok{PV_no_cc =} \KeywordTok{PV}\NormalTok{(U_no_cc, }\KeywordTok{ramsey}\NormalTok{(delta_no_cc, eta_no_cc, g_no_cc), t))}
  
\NormalTok{PV_BAU_reduce <-}\StringTok{ }\KeywordTok{sum}\NormalTok{(BAU_reduce}\OperatorTok{$}\NormalTok{PV)}

\NormalTok{df_temp <-}\StringTok{ }\KeywordTok{data.frame}\NormalTok{(}\DataTypeTok{theta =}\NormalTok{ i,}
                    \DataTypeTok{PV =} \KeywordTok{sum}\NormalTok{(BAU_reduce}\OperatorTok{$}\NormalTok{PV))}

\NormalTok{df_2d <-}\StringTok{ }\NormalTok{df_2d }\OperatorTok
\StringTok{  }\KeywordTok{full_join}\NormalTok{(df_temp)}
\NormalTok{\}}

\NormalTok{df_2d_sol <-}\StringTok{ }\NormalTok{df_2d }\OperatorTok
\StringTok{  }\KeywordTok{filter}\NormalTok{(}\KeywordTok{round}\NormalTok{(PV, }\DecValTok{4}\NormalTok{) }\OperatorTok{==}\StringTok{ }\KeywordTok{round}\NormalTok{(PV_uncertainty, }\DecValTok{4}\NormalTok{))}

\NormalTok{theta_star_}\DecValTok{2}\NormalTok{ <-}\StringTok{ }\KeywordTok{mean}\NormalTok{(df_2d_sol}\OperatorTok{$}\NormalTok{theta)}
\end{Highlighting}
\end{Shaded}

The maximum value society would be willing to endure every year to
completely prevent climate change with the uncertainty over T is theta*
= 0.3779.

\end{document}
